\documentclass[]{article}
\usepackage{lmodern}
\usepackage{amssymb,amsmath}
\usepackage{ifxetex,ifluatex}
\usepackage{fixltx2e} % provides \textsubscript
\ifnum 0\ifxetex 1\fi\ifluatex 1\fi=0 % if pdftex
  \usepackage[T1]{fontenc}
  \usepackage[utf8]{inputenc}
\else % if luatex or xelatex
  \ifxetex
    \usepackage{mathspec}
  \else
    \usepackage{fontspec}
  \fi
  \defaultfontfeatures{Ligatures=TeX,Scale=MatchLowercase}
\fi
% use upquote if available, for straight quotes in verbatim environments
\IfFileExists{upquote.sty}{\usepackage{upquote}}{}
% use microtype if available
\IfFileExists{microtype.sty}{%
\usepackage{microtype}
\UseMicrotypeSet[protrusion]{basicmath} % disable protrusion for tt fonts
}{}
\usepackage{hyperref}
\PassOptionsToPackage{usenames,dvipsnames}{color} % color is loaded by hyperref
\hypersetup{unicode=true,
            pdftitle={Yet Another Markdown Cheatsheet},
            pdfauthor={Kolen Cheung},
            colorlinks=true,
            linkcolor=Maroon,
            citecolor=blue,
            urlcolor=blue,
            breaklinks=true}
\urlstyle{same}  % don't use monospace font for urls
\usepackage{color}
\usepackage{fancyvrb}
\newcommand{\VerbBar}{|}
\newcommand{\VERB}{\Verb[commandchars=\\\{\}]}
\DefineVerbatimEnvironment{Highlighting}{Verbatim}{commandchars=\\\{\}}
% Add ',fontsize=\small' for more characters per line
\newenvironment{Shaded}{}{}
\newcommand{\KeywordTok}[1]{\textcolor[rgb]{0.00,0.44,0.13}{\textbf{{#1}}}}
\newcommand{\DataTypeTok}[1]{\textcolor[rgb]{0.56,0.13,0.00}{{#1}}}
\newcommand{\DecValTok}[1]{\textcolor[rgb]{0.25,0.63,0.44}{{#1}}}
\newcommand{\BaseNTok}[1]{\textcolor[rgb]{0.25,0.63,0.44}{{#1}}}
\newcommand{\FloatTok}[1]{\textcolor[rgb]{0.25,0.63,0.44}{{#1}}}
\newcommand{\ConstantTok}[1]{\textcolor[rgb]{0.53,0.00,0.00}{{#1}}}
\newcommand{\CharTok}[1]{\textcolor[rgb]{0.25,0.44,0.63}{{#1}}}
\newcommand{\SpecialCharTok}[1]{\textcolor[rgb]{0.25,0.44,0.63}{{#1}}}
\newcommand{\StringTok}[1]{\textcolor[rgb]{0.25,0.44,0.63}{{#1}}}
\newcommand{\VerbatimStringTok}[1]{\textcolor[rgb]{0.25,0.44,0.63}{{#1}}}
\newcommand{\SpecialStringTok}[1]{\textcolor[rgb]{0.73,0.40,0.53}{{#1}}}
\newcommand{\ImportTok}[1]{{#1}}
\newcommand{\CommentTok}[1]{\textcolor[rgb]{0.38,0.63,0.69}{\textit{{#1}}}}
\newcommand{\DocumentationTok}[1]{\textcolor[rgb]{0.73,0.13,0.13}{\textit{{#1}}}}
\newcommand{\AnnotationTok}[1]{\textcolor[rgb]{0.38,0.63,0.69}{\textbf{\textit{{#1}}}}}
\newcommand{\CommentVarTok}[1]{\textcolor[rgb]{0.38,0.63,0.69}{\textbf{\textit{{#1}}}}}
\newcommand{\OtherTok}[1]{\textcolor[rgb]{0.00,0.44,0.13}{{#1}}}
\newcommand{\FunctionTok}[1]{\textcolor[rgb]{0.02,0.16,0.49}{{#1}}}
\newcommand{\VariableTok}[1]{\textcolor[rgb]{0.10,0.09,0.49}{{#1}}}
\newcommand{\ControlFlowTok}[1]{\textcolor[rgb]{0.00,0.44,0.13}{\textbf{{#1}}}}
\newcommand{\OperatorTok}[1]{\textcolor[rgb]{0.40,0.40,0.40}{{#1}}}
\newcommand{\BuiltInTok}[1]{{#1}}
\newcommand{\ExtensionTok}[1]{{#1}}
\newcommand{\PreprocessorTok}[1]{\textcolor[rgb]{0.74,0.48,0.00}{{#1}}}
\newcommand{\AttributeTok}[1]{\textcolor[rgb]{0.49,0.56,0.16}{{#1}}}
\newcommand{\RegionMarkerTok}[1]{{#1}}
\newcommand{\InformationTok}[1]{\textcolor[rgb]{0.38,0.63,0.69}{\textbf{\textit{{#1}}}}}
\newcommand{\WarningTok}[1]{\textcolor[rgb]{0.38,0.63,0.69}{\textbf{\textit{{#1}}}}}
\newcommand{\AlertTok}[1]{\textcolor[rgb]{1.00,0.00,0.00}{\textbf{{#1}}}}
\newcommand{\ErrorTok}[1]{\textcolor[rgb]{1.00,0.00,0.00}{\textbf{{#1}}}}
\newcommand{\NormalTok}[1]{{#1}}
\usepackage{fancyvrb}
\VerbatimFootnotes % allows verbatim text in footnotes
\usepackage{longtable,booktabs}
\usepackage{graphicx,grffile}
\makeatletter
\def\maxwidth{\ifdim\Gin@nat@width>\linewidth\linewidth\else\Gin@nat@width\fi}
\def\maxheight{\ifdim\Gin@nat@height>\textheight\textheight\else\Gin@nat@height\fi}
\makeatother
% Scale images if necessary, so that they will not overflow the page
% margins by default, and it is still possible to overwrite the defaults
% using explicit options in \includegraphics[width, height, ...]{}
\setkeys{Gin}{width=\maxwidth,height=\maxheight,keepaspectratio}
\usepackage[normalem]{ulem}
% avoid problems with \sout in headers with hyperref:
\pdfstringdefDisableCommands{\renewcommand{\sout}{}}
\IfFileExists{parskip.sty}{%
\usepackage{parskip}
}{% else
\setlength{\parindent}{0pt}
\setlength{\parskip}{6pt plus 2pt minus 1pt}
}
\setlength{\emergencystretch}{3em}  % prevent overfull lines
\providecommand{\tightlist}{%
  \setlength{\itemsep}{0pt}\setlength{\parskip}{0pt}}
\setcounter{secnumdepth}{5}
% Redefines (sub)paragraphs to behave more like sections
\ifx\paragraph\undefined\else
\let\oldparagraph\paragraph
\renewcommand{\paragraph}[1]{\oldparagraph{#1}\mbox{}}
\fi
\ifx\subparagraph\undefined\else
\let\oldsubparagraph\subparagraph
\renewcommand{\subparagraph}[1]{\oldsubparagraph{#1}\mbox{}}
\fi

\title{Yet Another Markdown Cheatsheet}
\providecommand{\subtitle}[1]{}
\subtitle{Including MarkDown, MultiMarkdown, pandoc, GFM and LaTeX Math Syntax by
MathJax}
\author{Kolen Cheung}
\date{}

\begin{document}
\maketitle

{
\hypersetup{linkcolor=blue}
\setcounter{tocdepth}{6}
\tableofcontents
}
\section{Contents}\label{contents}

\{:.no\_toc\}

\begin{itemize}
\tightlist
\item
  Will be replaced with the ToC, excluding the ``Contents'' header
  \{:toc\}
\end{itemize}

\{\{TOC\}\}

\section{Introduction}\label{introduction}

\protect\hyperlink{examples}{Examples} shows explicit examples for
different syntaxes. \protect\hyperlink{other-syntaxes}{Other Syntaxes}
show the syntaxes that can't be shown explicitly.

\subsection{Organization}\label{organization}

\begin{itemize}
\tightlist
\item
  Header levels (except possibly the last header level): features in
  groups
\item
  Last header level or a list: different syntaxes
\item
  TaskPaper-styled tags to indicate in what favor of Markdown such
  syntax is supported

  \begin{itemize}
  \tightlist
  \item
    \texttt{@markdown}: supported by original markdown, hence understood
    to be supported by all variants of markdown
  \item
    \texttt{@ghpages}: GitHub-Favored Markdown, built by kramdown with
    GFM option. i.e.~GitHub Pages' GitHub-Favored Markdown
  \item
    \texttt{@mmd}: MultiMarkdown~
  \item
    \texttt{@pandoc}: pandoc-favored markdown
  \item
    \texttt{@phpextra}: PHP Markdown Extra (inspired some syntax in
    pandoc and mmd and gfm, not exhaustively tested here)
  \end{itemize}
\item
  TaskPaper Tags

  \begin{itemize}
  \tightlist
  \item
    \texttt{@...(partial)}: partial supports only
  \item
    \texttt{@...(+...)}: when the extension is used
  \item
    \texttt{@pandoc(-\/-...)}: when the command line argument is used
  \item
    \texttt{@pandoc(parsed)}: not verbatim, but parsed
  \end{itemize}
\end{itemize}

Note:

\begin{itemize}
\tightlist
\item
  You might see
  \texttt{\textless{}!-\/-\ \textbackslash{}begin\{comment\}\ -\/-\textgreater{}...\textless{}!-\/-\ \textbackslash{}end\{comment\}\ -\/-\textgreater{}}.
  This is for mmd to tex to pdf use only. Ignore this.
\end{itemize}

\hypertarget{examples}{\section{Examples}\label{examples}}

\hypertarget{header}{\subsection{Header}\label{header}}

@markdown

See \protect\hyperlink{emphasis}{Emphasis} and
\protect\hyperlink{other-syntaxes}{Other Syntaxes} to see alternative
Setext-style header styles @markdown

\subsubsection{\texorpdfstring{Header \emph{Containing}
\textbf{\emph{Styling}} and a
\href{Https://www.wikipedia.org/}{Link}}{Header Containing Styling and a Link}}\label{header-containing-styling-and-a-link}

@markdown

\hypertarget{identifier}{\subsubsection{Header Containing
Attributes}\label{identifier}}

@pandoc @phpextra

\subsubsection*{Header Unnumbered}\label{header-unnumbered}
\addcontentsline{toc}{subsubsection}{Header Unnumbered}

@pandoc

\subsubsection{Header Unnumbered 2}\label{header-unnumbered-2}

@pandoc

\subsubsection{Auto Cross Reference}\label{auto-cross-reference}

\begin{itemize}
\tightlist
\item
  \protect\hyperlink{header}{Link to Header} @pandoc @ghpages @mmd
\item
  \protect\hyperlink{header}{Link to Header} @pandoc @mmd
\item
  \protect\hyperlink{header}{Header} @mmd @pandoc
\item
  \protect\hyperlink{header}{Header} @mmd @pandoc
\end{itemize}

\hypertarget{userdefinedreference}{\subsubsection{User defined
reference}\label{userdefinedreference}}

\begin{itemize}
\tightlist
\item
  {[}userdefinedreference{]}{[}{]} @mmd
\item
  \protect\hyperlink{userdefinedreference}{Link to userdefinedreference}
  @mmd @pandoc(+mmd\_header\_identifiers)
\item
  \protect\hyperlink{identifier}{Link to ``Header Containing
  Attributes''} @pandoc
\item
  \protect\hyperlink{identifier}{another-link} @pandoc
\end{itemize}

\subsubsection{Deeper Levels of Headers}\label{deeper-levels-of-headers}

\paragraph{Header4}\label{header4}

\subparagraph{Header5}\label{header5}

Header6

\subsection{Backslash Escapes}\label{backslash-escapes}

*testing* @markdown

\hypertarget{emphasis}{\subsection{Emphasis}\label{emphasis}}

\begin{itemize}
\tightlist
\item
  \emph{italic} or \emph{italic} @markdown
\item
  \textbf{bold} or \textbf{bold} @markdown
\item
  \textbf{\emph{bold italic}} or \textbf{\emph{bold italic}} @markdown
\item
  \sout{strikethrough} @pandoc
\item
  \textsc{Small caps} @pandoc @markdown(html)
\end{itemize}

\subsubsection{CriticMarkup}\label{criticmarkup}

Visually it looks like emphasis. Functionally it is much more, and
called Critic Markup @mmd

\begin{itemize}
\tightlist
\item
  Deletions from the original text: This is \{--is --\}a test.
\item
  Additions: This \{++is ++\}a test.
\item
  Substitutions: This
  \{\textsubscript{\textsubscript{isn't}\textgreater{}is}\textasciitilde{}\}
  a test.
\item
  Highlighting: This is a \{==test==\}.
\item
  Comments: This is a test\{\textgreater{}\textgreater{}What is it a
  test of?\textless{}\textless{}\}.
\end{itemize}

See more at
\href{http://fletcher.github.io/MultiMarkdown-5/criticmarkup.html}{CriticMarkup---MultiMarkdown
Documentation}.

\subsection{Horizontal Rules}\label{horizontal-rules}

@markdown

\begin{center}\rule{0.5\linewidth}{\linethickness}\end{center}

3 or more hyphens or asterisks

\begin{center}\rule{0.5\linewidth}{\linethickness}\end{center}

\subsection{Break}\label{break}

@markdown

No break like this

Soft break\\
like this

Hard break

like this

\subsection{Superscript \& Subscript}\label{superscript-subscript}

\begin{itemize}
\tightlist
\item
  x\^{}2 @mmd
\item
  d\textasciitilde{}o @mmd
\item
  x\textsuperscript{a+b} @mmd @pandoc
\item
  x\textsubscript{y,z} @mmd @pandoc
\item
  P\textsubscript{a~cat} @pandoc
\end{itemize}

\subsection{Smarty Pants}\label{smarty-pants}

@markdown(+smartypants) @pandoc(--smart) @ghpages

\begin{itemize}
\tightlist
\item
  ``Example 1''
\item
  `Example 2'
\item
  en--dash
\item
  em---dash
\item
  ellipsis\ldots{}
\end{itemize}

@mmd

\begin{itemize}
\tightlist
\item
  ``Example 3''
\end{itemize}

\subsection{Abbreviations (PHP Markdown
Extra)}\label{abbreviations-php-markdown-extra}

@mmd @phpextra @pandoc(+abbreviations)

Testing abbreviations: HTML, W3C (mouseover it to see)

\subsection{Lists}\label{lists}

\subsubsection{Ordered Lists}\label{ordered-lists}

@markdown

\begin{enumerate}
\def\labelenumi{\arabic{enumi}.}
\tightlist
\item
  test
\item
  test
\item
  test
\end{enumerate}

\subsubsection{Unordered Lists}\label{unordered-lists}

@markdown

\begin{itemize}
\tightlist
\item
  test
\item
  test
\item
  test
\end{itemize}

\subsubsection{Nested Lists}\label{nested-lists}

@markdown

\begin{itemize}
\tightlist
\item
  test

  \begin{itemize}
  \tightlist
  \item
    test
  \end{itemize}
\item
  test

  \begin{enumerate}
  \def\labelenumi{\arabic{enumi}.}
  \tightlist
  \item
    test
  \item
    test

    \begin{itemize}
    \tightlist
    \item
      test

      \begin{enumerate}
      \def\labelenumii{\arabic{enumii}.}
      \tightlist
      \item
        test
      \item
        test
      \end{enumerate}
    \end{itemize}
  \item
    test
  \end{enumerate}
\item
  test
\end{itemize}

Note about LaTeX output in mmd/pandoc:

\begin{itemize}
\tightlist
\item
  The Maximum nesting level of lists in LaTeX is 4. The quick hack is to
  mix itemize and enumerate alternatively to go beyond this.
\end{itemize}

\subsubsection{Cutoff a List}\label{cutoff-a-list}

@markdown

\begin{enumerate}
\def\labelenumi{\arabic{enumi}.}
\tightlist
\item
  one
\item
  two
\item
  three
\end{enumerate}

\begin{enumerate}
\def\labelenumi{\arabic{enumi}.}
\tightlist
\item
  uno
\item
  dos
\item
  tres
\end{enumerate}

\begin{itemize}
\tightlist
\item
  item one
\item
  item two
\end{itemize}

\begin{verbatim}
{ my code block }
\end{verbatim}

\subsubsection{List Item in a Block}\label{list-item-in-a-block}

@markdown

\begin{itemize}
\item
  First paragraph.

  Continued.
\item
  Second paragraph. With a code block, which must be indented eight
  spaces:

\begin{verbatim}
{ code }
\end{verbatim}
\end{itemize}

\subsubsection{Fancy Lists}\label{fancy-lists}

@pandoc

\begin{enumerate}
\tightlist
\item
  one
\item
  two
\end{enumerate}

\begin{enumerate}
\def\labelenumi{\arabic{enumi})}
\setcounter{enumi}{8}
\tightlist
\item
  Ninth
\item
  Tenth
\item
  Eleventh

  \begin{enumerate}
  \def\labelenumii{\roman{enumii}.}
  \tightlist
  \item
    \texttt{i}
  \item
    \texttt{ii}
  \item
    \texttt{iii}
  \end{enumerate}
\end{enumerate}

\begin{enumerate}
\def\labelenumi{(\arabic{enumi})}
\setcounter{enumi}{1}
\tightlist
\item
  Two
\item
  Three
\end{enumerate}

\begin{enumerate}
\def\labelenumi{\arabic{enumi}.}
\tightlist
\item
  Four
\end{enumerate}

\begin{itemize}
\tightlist
\item
  Five
\end{itemize}

\subsection{Definition Lists}\label{definition-lists}

\subsubsection{Method 1}\label{method-1}

@mmd @phpextra @pandoc @ghpages

\begin{description}
\tightlist
\item[Physics]
The Fundamental of Science

Describe the Nature

Make Prediction
\end{description}

\subsubsection{Method 2}\label{method-2}

@ghpages @pandoc @mmd

\begin{description}
\item[Term 1]
Definition 1
\item[Term 2 with \emph{inline markup}]
Definition 2

\begin{verbatim}
{ some code, part of Definition 2 }
\end{verbatim}

Third paragraph of definition 2.
\end{description}

\subsection{Numbered Example Lists}\label{numbered-example-lists}

@pandoc

\begin{enumerate}
\def\labelenumi{(\arabic{enumi})}
\tightlist
\item
  My first example will be numbered (1).
\item
  My second example will be numbered (2).
\end{enumerate}

Explanation of examples.

\begin{enumerate}
\def\labelenumi{(\arabic{enumi})}
\setcounter{enumi}{2}
\item
  My third example will be numbered (3).
\item
  This is a good example.
\end{enumerate}

As (4) illustrates, \ldots{}

\subsection{Code}\label{code}

\begin{itemize}
\tightlist
\item
  \texttt{testing} @markdown
\item
  \VERB|\NormalTok{\textbackslash{}[\textbackslash{}ket\{a\}\textbackslash{}]}|
  @pandoc
\end{itemize}

\subsubsection{Fenced Code Blocks}\label{fenced-code-blocks}

\paragraph{Method 1}\label{method-1-1}

@markdown

\begin{verbatim}
test
test
    test
    # test
\end{verbatim}

\paragraph{Method 2}\label{method-2-1}

@markdown(partial:language-not-supported) @ghpages @pandoc @mmd

\begin{Shaded}
\begin{Highlighting}[]
\NormalTok{\textbackslash{}nabla \textbackslash{}times \textbackslash{}mathbf\{E\} = - \textbackslash{}frac\{\textbackslash{}partial \textbackslash{}mathbf\{B\}\}\{\textbackslash{}partial t\}}
\end{Highlighting}
\end{Shaded}

\paragraph{Method 3}\label{method-3}

@pandoc

\begin{Shaded}
\begin{Highlighting}[]
\NormalTok{test}
\NormalTok{test}
\BaseNTok{    test}
\BaseNTok{    # test}
\end{Highlighting}
\end{Shaded}

\hypertarget{mycode}{\label{mycode}}
\begin{Shaded}
\begin{Highlighting}[numbers=left,,firstnumber=100,]
\NormalTok{test}
\NormalTok{test}
\BaseNTok{    test}
\BaseNTok{    # test}
\end{Highlighting}
\end{Shaded}

\subsection{Block-quotes}\label{block-quotes}

@markdown

\begin{quote}
\mbox{}%
\paragraph{Test}\label{test}

test

\begin{quote}
test
\end{quote}
\end{quote}

\begin{quote}
\begin{quote}
test
\end{quote}

\begin{itemize}
\tightlist
\item
  test
\end{itemize}
\end{quote}

\begin{quote}
\begin{itemize}
\tightlist
\item
  test
\end{itemize}
\end{quote}

\subsubsection{Block-quotes Quoting
Codes}\label{block-quotes-quoting-codes}

@markdown

\begin{quote}
\begin{verbatim}
\newcommand...
\end{verbatim}
\end{quote}

\subsection{Line Blocks}\label{line-blocks}

@ghpages(partial) @pandoc

The limerick packs laughs anatomical\\
In space that is quite economical.\\
\hspace*{0.333em}\hspace*{0.333em}\hspace*{0.333em}But the good ones
I've seen\\
\hspace*{0.333em}\hspace*{0.333em}\hspace*{0.333em}So seldom are clean\\
And the clean ones so seldom are comical

200 Main St.\\
Berkeley, CA 94718

\subsection{Tables}\label{tables}

\subsubsection{Method 1}\label{method-1-2}

@ghpages @pandoc @mmd

\begin{longtable}[]{@{}rllc@{}}
\caption{Notice how pandoc can have optional caption like
this}\tabularnewline
\toprule
Right & Left & Default & Center\tabularnewline
\midrule
\endfirsthead
\toprule
Right & Left & Default & Center\tabularnewline
\midrule
\endhead
12 & 12 & 12 & 12\tabularnewline
123 & 123 & 123 & 123\tabularnewline
1 & 1 & 1 & 1\tabularnewline
\bottomrule
\end{longtable}

\subsubsection{Method 2}\label{method-2-2}

@mmd

{[}Table Caption{]} \textbar{} \textbar{} Grouping \textbar{}\textbar{}
\textbar{} Left align \textbar{} Right align \textbar{} Center align
\textbar{}
\textbar{}:-----------\textbar{}------------:\textbar{}:------------:\textbar{}
\textbar{} This \textbar{} This \textbar{} This \textbar{} \textbar{}
column \textbar{} column \textbar{} column \textbar{} \textbar{} will
\textbar{} will \textbar{} will \textbar{} \textbar{} be \textbar{} be
\textbar{} be \textbar{} \textbar{} left \textbar{} right \textbar{}
center \textbar{} \textbar{} aligned \textbar{} aligned \textbar{}
aligned \textbar{}\\
\textbar{} And a big grouping is like this
\textbar{}\textbar{}\textbar{}

\subsubsection{Method 3}\label{method-3-1}

@pandoc

\begin{longtable}[]{@{}rlcl@{}}
\caption{Demonstration of simple table syntax.}\tabularnewline
\toprule
Right & Left & Center & Default\tabularnewline
\midrule
\endfirsthead
\toprule
Right & Left & Center & Default\tabularnewline
\midrule
\endhead
12 & 12 & 12 & 12\tabularnewline
123 & 123 & 123 & 123\tabularnewline
1 & 1 & 1 & 1\tabularnewline
\bottomrule
\end{longtable}

\subsubsection{Method 4}\label{method-4}

@pandoc

\begin{longtable}[]{@{}clrl@{}}
\caption{Here's the caption. It, too, may span multiple
lines.}\tabularnewline
\toprule
\begin{minipage}[b]{0.15\columnwidth}\centering\strut
Centered Header
\strut\end{minipage} &
\begin{minipage}[b]{0.10\columnwidth}\raggedright\strut
Default Aligned
\strut\end{minipage} &
\begin{minipage}[b]{0.20\columnwidth}\raggedleft\strut
Right Aligned
\strut\end{minipage} &
\begin{minipage}[b]{0.31\columnwidth}\raggedright\strut
Left Aligned
\strut\end{minipage}\tabularnewline
\midrule
\endfirsthead
\toprule
\begin{minipage}[b]{0.15\columnwidth}\centering\strut
Centered Header
\strut\end{minipage} &
\begin{minipage}[b]{0.10\columnwidth}\raggedright\strut
Default Aligned
\strut\end{minipage} &
\begin{minipage}[b]{0.20\columnwidth}\raggedleft\strut
Right Aligned
\strut\end{minipage} &
\begin{minipage}[b]{0.31\columnwidth}\raggedright\strut
Left Aligned
\strut\end{minipage}\tabularnewline
\midrule
\endhead
\begin{minipage}[t]{0.15\columnwidth}\centering\strut
First
\strut\end{minipage} &
\begin{minipage}[t]{0.10\columnwidth}\raggedright\strut
row
\strut\end{minipage} &
\begin{minipage}[t]{0.20\columnwidth}\raggedleft\strut
12.0
\strut\end{minipage} &
\begin{minipage}[t]{0.31\columnwidth}\raggedright\strut
Example of a row that spans multiple lines.
\strut\end{minipage}\tabularnewline
\begin{minipage}[t]{0.15\columnwidth}\centering\strut
Second
\strut\end{minipage} &
\begin{minipage}[t]{0.10\columnwidth}\raggedright\strut
row
\strut\end{minipage} &
\begin{minipage}[t]{0.20\columnwidth}\raggedleft\strut
5.0
\strut\end{minipage} &
\begin{minipage}[t]{0.31\columnwidth}\raggedright\strut
Here's another one. Note the blank line between rows.
\strut\end{minipage}\tabularnewline
\bottomrule
\end{longtable}

\subsubsection{Method 5}\label{method-5}

@pandoc

\begin{longtable}[]{@{}lll@{}}
\caption{Sample grid table.}\tabularnewline
\toprule
\begin{minipage}[b]{0.20\columnwidth}\raggedright\strut
Fruit
\strut\end{minipage} &
\begin{minipage}[b]{0.20\columnwidth}\raggedright\strut
Price
\strut\end{minipage} &
\begin{minipage}[b]{0.27\columnwidth}\raggedright\strut
Advantages
\strut\end{minipage}\tabularnewline
\midrule
\endfirsthead
\toprule
\begin{minipage}[b]{0.20\columnwidth}\raggedright\strut
Fruit
\strut\end{minipage} &
\begin{minipage}[b]{0.20\columnwidth}\raggedright\strut
Price
\strut\end{minipage} &
\begin{minipage}[b]{0.27\columnwidth}\raggedright\strut
Advantages
\strut\end{minipage}\tabularnewline
\midrule
\endhead
\begin{minipage}[t]{0.20\columnwidth}\raggedright\strut
Bananas
\strut\end{minipage} &
\begin{minipage}[t]{0.20\columnwidth}\raggedright\strut
\$1.34
\strut\end{minipage} &
\begin{minipage}[t]{0.27\columnwidth}\raggedright\strut
\begin{itemize}
\tightlist
\item
  built-in wrapper
\item
  bright color
\end{itemize}
\strut\end{minipage}\tabularnewline
\begin{minipage}[t]{0.20\columnwidth}\raggedright\strut
Oranges
\strut\end{minipage} &
\begin{minipage}[t]{0.20\columnwidth}\raggedright\strut
\$2.10
\strut\end{minipage} &
\begin{minipage}[t]{0.27\columnwidth}\raggedright\strut
\begin{itemize}
\tightlist
\item
  cures scurvy
\item
  tasty
\end{itemize}
\strut\end{minipage}\tabularnewline
\bottomrule
\end{longtable}

\subsubsection{Method 6}\label{method-6}

@ghpages

\textbar{}-----------------+------------+-----------------+----------------\textbar{}
\textbar{} Default aligned \textbar{}Left aligned\textbar{} Center
aligned \textbar{} Right aligned \textbar{}
\textbar{}-----------------\textbar{}:-----------\textbar{}:---------------:\textbar{}---------------:\textbar{}
\textbar{} First body part \textbar{}Second cell \textbar{} Third cell
\textbar{} fourth cell \textbar{} \textbar{} Second line \textbar{}foo
\textbar{} \textbf{strong} \textbar{} blah \textbar{} \textbar{} Third
line \textbar{}blah \textbar{} blah \textbar{} bar \textbar{}
\textbar{}-----------------+------------+-----------------+----------------\textbar{}
\textbar{} Second body \textbar{} \textbar{} \textbar{} \textbar{}
\textbar{} 2 line \textbar{} \textbar{} \textbar{} \textbar{}
\textbar{}=================+============+=================+================\textbar{}
\textbar{} Footer row \textbar{} \textbar{} \textbar{} \textbar{}
\textbar{}-----------------+------------+-----------------+----------------\textbar{}

See \href{http://kramdown.gettalong.org/syntax.html\#tables}{Syntax
\textbar{} kramdown}

\subsection{Links}\label{links}

\begin{itemize}
\tightlist
\item
  \url{https://www.wikipedia.org} @pandoc(+autolink\_bare\_uris)
\item
  \url{https://www.wikipedia.org} @markdown
\item
  \href{https://www.wikipedia.org}{Wikipedia.org} @markdown
\item
  \href{https://www.wikipedia.org}{Wikipedia.org} @markdown
\item
  \href{mailto:support@github.com}{Mail to GitHub} @markdown
\end{itemize}

\subsubsection{Reference Links}\label{reference-links}

\begin{itemize}
\tightlist
\item
  \href{https://www.google.com}{Search here} @markdown
\item
  A \href{image.png}{link} with attributes. @mmd
\item
  \href{image.png}{Another link to the link above}. @mmd
\item
  A \href{/foo/bar.html}{link1} with attributes. @markdown
\item
  \href{http://fsf.org}{Another link}. @markdown
\item
  \href{/bar\#special}{link3}. @markdown
\end{itemize}

\subsection{Footnotes}\label{footnotes}

\begin{itemize}
\tightlist
\item
  Footnotes {[}\^{}This is a mmd inline footnote{]} @mmd
\item
  Footnotes \footnote{This is a pandoc inline footnote} @pandoc
\end{itemize}

\subsubsection{Reference Footnotes}\label{reference-footnotes}

\begin{itemize}
\tightlist
\item
  Footnotes\footnote{This is a footnote} @markdown
\item
  Long Footnotes \footnote{Here's one with multiple blocks.

    Subsequent paragraphs are indented to show that they belong to the
    previous footnote.

\begin{Verbatim}
{ some.code }
\end{Verbatim}

    The whole paragraph can be indented, or just the first line. In this
    way, multi-paragraph footnotes work like multi-paragraph list items.}
  @pandoc @ghpages @mmd
\end{itemize}

This paragraph won't be part of the note, because it isn't indented.

\subsubsection{Glossaries}\label{glossaries}

A special kind of footnote \footnote{glossary: term (optional sort key)
  The actual definition belongs on a new line, and can continue on just
  as other footnotes. Note how it fallbacks gracefully in Markdown.}.
@mmd

See more at
\href{http://fletcher.github.io/MultiMarkdown-5/glossary.html}{Glossary---MultiMarkdown
Documentation}.

\subsubsection{Citations}\label{citations}

It can looks like footnotes in HTML output.

\paragraph{MultiMarkdown}\label{multimarkdown}

@mmd

\begin{itemize}
\tightlist
\item
  This is a statement that should be attributed to its source
  \href{John\%20Doe.\%20*Some\%20Big\%20Fancy\%20Book*.\%20Vanity\%20Press,\%202006.}{p.~23}.
\end{itemize}

\href{John\%20Doe.\%20*Another\%20Big\%20Fancy\%20Book*.\%20Vanity\%20Press,\%202016.}{Not
cited}

See more at
\href{http://fletcher.github.io/MultiMarkdown-5/citations.html}{Citations---MultiMarkdown
Documentation}.

\paragraph{Pandoc}\label{pandoc}

@pandoc

Very powerful but complicated. See more at
\href{http://pandoc.org/README.html\#citations}{Citations---Pandoc
Documentation}.

\subsection{Images}\label{images}

\begin{itemize}
\tightlist
\item
  \includegraphics{image.png} @markdown
\end{itemize}

\subsubsection{Reference Images}\label{reference-images}

\begin{itemize}
\tightlist
\item
  \includegraphics{image.png} @markdown
\item
  \includegraphics[width=0.41667in,height=0.41667in]{image.png} @mmd
  @pandoc(+mmd\_link\_attributes)
\item
  \includegraphics[width=0.31250in,height=0.20833in]{image.png} @pandoc
  @phpextra(partial)
\item
  a reference \includegraphics{image.png} with attributes. @pandoc
  @phpextra(partial)
\end{itemize}

\subsubsection{Image with Links by Nesting Image and
Link}\label{image-with-links-by-nesting-image-and-link}

\begin{itemize}
\tightlist
\item
  \href{https://www.wikipedia.org/}{\includegraphics{image.png}}
  @markdown
\end{itemize}

\subsubsection{Block Level Images}\label{block-level-images}

\begin{itemize}
\tightlist
\item
  Block level: HTML \texttt{figure} element in MultiMarkdown @mmd
  @pandoc
\end{itemize}

\begin{figure}[htbp]
\centering
\includegraphics{image.png}
\caption{\textbf{\emph{Block}} \textbf{Level}}
\end{figure}

\includegraphics{image.png}\\
\#\# RAW \#\#

\subsubsection{HTML}\label{html}

\begin{itemize}
\item
  This should \emph{not} be markdown (or is it?)

  @markdown
\item
  This \emph{is} markdown

  @mmd @pandoc(+markdown\_attribute)
\end{itemize}

See more at
\href{http://fletcher.github.io/MultiMarkdown-5/raw.html}{Raw---MultiMarkdown
Documentation}. See test in {[}Babelmark 2 - Compare markdown
implementations{]}(\url{http://johnmacfarlane.net/babelmark2/?normalize=1\&text=\%3Cdiv\%3EThis+should+*not*+be+markdown+(or+is+it\%3F\%29+\%3C\%2Fdiv\%3E\%0A\%3Cdiv+markdown\%3D1\%3EThis+*is*+markdown\%3C\%2Fdiv\%3E)}.

\subsubsection{LaTeX}\label{latex}

\begin{itemize}
\item
  @mmd
\item
  \newcommand\rawlatex{} 

  @pandoc(parsed)
\item
  \textbackslash{}begin\{\ldots{}\} @pandoc
\end{itemize}

\hypertarget{other-syntaxes}{\section{Other
Syntaxes}\label{other-syntaxes}}

\subsection{Metadata}\label{metadata}

Note: mmd accepts capitalized metadata keys but others do not. For
maximum compatibility, \texttt{author(s)}, \texttt{title}, etc. should
be in lower cases.

\subsubsection{MultiMarkdown Metadata
Block}\label{multimarkdown-metadata-block}

@mmd @pandoc(+mmd\_title\_block)

\begin{verbatim}
title:    A Sample MultiMarkdown Document  
author:   Fletcher T. Penney  
date:     February 9, 2011  
comment:  This is a comment intended to demonstrate  
          metadata that spans multiple lines, yet  
          is treated as a single value.  
CSS:      http://example.com/standard.css
\end{verbatim}

See more at
\href{http://fletcher.github.io/MultiMarkdown-5/metadata.html}{Metadata---MultiMarkdown
Documentation}.

\subsubsection{Pandoc Title Block}\label{pandoc-title-block}

@pandoc

\begin{verbatim}
% title
% author(s) (separated by semicolons)
% date
\end{verbatim}

\subsubsection{YAML Metadata Block}\label{yaml-metadata-block}

@Mmd(partial)

@pandoc @ghpages

\begin{verbatim}
---
title:    A Sample MultiMarkdown Document  
author:   Fletcher T. Penney  
date:     February 9, 2011  
tags: [nothing, nothingness]
abstract: |
  This is the abstract.

  It consists of two paragraphs.
---
\end{verbatim}

\subsection{TOC}\label{toc}

\subsubsection{Pandoc}\label{pandoc-1}

@pandoc

Use \texttt{-\/-toc} as a command argument.

\subsubsection{MultiMarkdown}\label{multimarkdown-1}

@mmd

\texttt{\{\{TOC\}\}}, see beginning. It preprocess the headings and
generate a ToC on its own, and doesn't give instruction for LaTeX to
generate one. A hack is like this:

\begin{verbatim}
---
...
LaTeX Input:    mmd-load-toc-setcounter
LaTeX Input:    mmd-load-toc
...
---
<!-- \begin{comment} -->
{{TOC}}
<!-- \end{comment} -->
...
\end{verbatim}

See more at
\href{https://github.com/ickc/peg-multimarkdown-latex-support}{ickc/peg-multimarkdown-latex-support:
Default support files for generating LaTeX documents with MMD 3 through
MMD 5}.

\subsubsection{Kramdown}\label{kramdown}

@ghpages

\begin{verbatim}
# Contents
{:.no_toc}

* Will be replaced with the ToC, excluding the "Contents" header
{:toc}
\end{verbatim}

\subsection{Math}\label{math}

MathJax is assumed. MathJax has many configurable options. See
\href{http://mathjax.readthedocs.org/en/latest/tex.html}{MathJax TeX and
LaTeX Support --- MathJax 2.6 documentation}.

\subsubsection{Markdown}\label{markdown}

@markdown

Add the following at the beginning of the document:

\begin{Shaded}
\begin{Highlighting}[]
\KeywordTok{<script}\OtherTok{ type=}\StringTok{"text/javascript"}
\OtherTok{    src=}\StringTok{"https://cdn.mathjax.org/mathjax/latest/MathJax.js?config=TeX-AMS_CHTML-full"}\KeywordTok{>}
    \KeywordTok{</script>}
\end{Highlighting}
\end{Shaded}

MathJax.js is used and any codes within math delimiters are treated as
raw HTML and to be processed by MathJax.

MathJax delimiter are \texttt{\$\$...\$\$},
\texttt{\textbackslash{}\textbackslash{}(...\textbackslash{}\textbackslash{})}
and
\texttt{\textbackslash{}\textbackslash{}{[}...\textbackslash{}\textbackslash{}{]}}
(because an extra \texttt{\textbackslash{}} can be used to escape from
MarkDown).

Depending on the markdown parser, extra tricks might be needed to make
sure nothing within the math delimiter is treated as markdown (see
\href{http://mathjax.readthedocs.org/en/latest/tex.html\#tex-and-latex-in-html-documents}{TeX
and LaTeX in HTML documents --- MathJax 2.6 documentation}).
\texttt{\$...\$} can be used with MathJax configuration (see
\href{http://mathjax.readthedocs.org/en/latest/tex.html\#tex-and-latex-math-delimiters}{TeX
and LaTeX math delimiters --- MathJax 2.6 documentation}).

\subsubsection{MultiMarkdown and Pandoc}\label{multimarkdown-and-pandoc}

There are subtleties how math should be used in HTML+MathJax and LaTeX
output from single markdown source. See more in
\href{https://github.com/ickc/mathjax-latex-md-mmd-pandoc}{Testing LaTeX
Environments Usage in MathJax From Markdown Conversion (including mmd
and pandoc)}.

\paragraph{MultiMarkdown}\label{multimarkdown-2}

@mmd

Add the following metadata at the beginning of the document:

\begin{Shaded}
\begin{Highlighting}[]
\NormalTok{HTML header:    }\KeywordTok{<script}\OtherTok{ type=}\StringTok{"text/javascript"}
\OtherTok{    src=}\StringTok{"https://cdn.mathjax.org/mathjax/latest/MathJax.js?config=TeX-AMS_CHTML-full"}\KeywordTok{>}
    \KeywordTok{</script>}
\end{Highlighting}
\end{Shaded}

MultiMarkdown math delimiter are \texttt{\$...\$}, \texttt{\$\$...\$\$},
\texttt{\textbackslash{}\textbackslash{}(...\textbackslash{}\textbackslash{})}
and
\texttt{\textbackslash{}\textbackslash{}{[}...\textbackslash{}\textbackslash{}{]}}.

\paragraph{Pandoc}\label{pandoc-2}

@pandoc(--mathjax)

For pandoc, add \texttt{-\/-mathjax} in the command-line argument.

Default math delimiter for pandoc is \texttt{\$...\$},
\texttt{\$\$...\$\$}. Other options are configurable. See more in
\href{http://pandoc.org/README.html\#non-pandoc-extensions}{Pandoc -
Pandoc User's Guide}.

\subsubsection{Inline Math}\label{inline-math}

\begin{itemize}
\tightlist
\item
  \(1+1\)
\item
  \(1 + 1\) @pandoc(+tex\_math\_double\_backslash)
\end{itemize}

\subsubsection{Block Math}\label{block-math}

\begin{itemize}
\tightlist
\item
  \[R R^T = I\]
\item
  \[A^T_S = B\]
\end{itemize}

\subsubsection{Other Examples}\label{other-examples}

\begin{itemize}
\tightlist
\item
  \[x = {-b \pm \sqrt{b^2-4ac} \over 2a}\]
\item
  \[
  \begin{aligned}
  \dot{x} & = \sigma(y-x) \\\
  \dot{y} & = \rho x - y - xz \\\
  \dot{z} & = -\beta z + xy
  \end{aligned}
  \]
\item
  \[\left( \sum_{k=1}^n a_k b_k \right)^2 \leq \left( \sum_{k=1}^n a_k^2 \right) \left( \sum_{k=1}^n b_k^2 \right)\]
\item
  \[\mathbf{V}_1 \times \mathbf{V}_2 =  \begin{vmatrix}
  \mathbf{i} & \mathbf{j} & \mathbf{k} \\\
  \frac{\partial X}{\partial u} &  \frac{\partial Y}{\partial u} & 0 \\\
  \frac{\partial X}{\partial v} &  \frac{\partial Y}{\partial v} & 0
  \end{vmatrix}\]
\item
  \[P(E) = {n \choose k} p^k (1-p)^{n-k}\]
\item
  \[\frac{1}{\Bigl(\sqrt{\phi \sqrt{5}}-\phi\Bigr) e^{\frac25 \pi}} =
  1+\frac{e^{-2\pi}} {1+\frac{e^{-4\pi}} {1+\frac{e^{-6\pi}}
  {1+\frac{e^{-8\pi}} {1+\ldots} } } }\]
\item
  \[
  \begin{aligned}
  \nabla \times \vec{\mathbf{B}} -\, \frac1c\, \frac{\partial\vec{\mathbf{E}}}{\partial t} & = \frac{4\pi}{c}\vec{\mathbf{j}} \\\
  \nabla \cdot \vec{\mathbf{E}} & = 4 \pi \rho \\\
  \nabla \times \vec{\mathbf{E}}\, +\, \frac1c\, \frac{\partial\vec{\mathbf{B}}}{\partial t} & = \vec{\mathbf{0}} \\\
  \nabla \cdot \vec{\mathbf{B}} & = 0 \end{aligned}
  \]
\item
  \[1 +  \frac{q^2}{(1-q)}+\frac{q^6}{(1-q)(1-q^2)}+\cdots =
  \prod_{j=0}^{\infty}\frac{1}{(1-q^{5j+2})(1-q^{5j+3})},
  \quad\quad \text{for $|q|<1$}.\]
\end{itemize}

\subsection{File Transclusion}\label{file-transclusion}

@mmd

See more at
\href{http://fletcher.github.io/MultiMarkdown-5/transclusion.html}{File
Transclusion---MultiMarkdown Documentation}.

\section{References}\label{references}

Some examples are directly or indirectly copied from the following
documentations:

\begin{enumerate}
\def\labelenumi{\arabic{enumi}.}
\tightlist
\item
  \href{http://pandoc.org/README.html}{Pandoc - Pandoc User's Guide}
\item
  \href{http://fletcher.github.io/MultiMarkdown-5/}{MultiMarkdown User's
  Guide}
\item
  \href{http://kramdown.gettalong.org/syntax.html\#tables}{Syntax
  \textbar{} kramdown}
\end{enumerate}

\end{document}
