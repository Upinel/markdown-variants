\def\htmlheaderlevel{1}
\input{mmd-article-header}
\input{mmd-natbib-plain}
\input{mmd-load-physics-related}
\input{mmd-load-tables-related}
\input{mmd-load-pdfpages}
\input{mmd-load-headings}
\def\thmd{chapter}
\input{mmd-load-amsthm}
\def\mytitle{Yet Another Markdown Cheatsheet}
\def\subtitle{Including MarkDown, MultiMarkdown, pandoc, GFM and LaTeX Math Syntax by MathJax}
\def\keywords{MarkDown, MultiMarkDown, LaTeX, pandoc, gfm}
\def\revision{0.10}
\def\language{English}
\def\myauthor{Kolen Cheung}
\def\email{khcheung@berkeley.edu}
\def\affiliation{University of California, Berkeley}
\def\latexmode{memoir}
\input{mmd-article-begin-doc}
\def\tocd{5}
\def\secd{5}
\input{mmd-load-toc-setcounter}
\input{mmd-load-toc}
\newacro{pandocpandocwillsimplyignorethisiftheextensionisused.agracefulfallback.}[PANDOC]{Pandoc will simply ignore this if the extension is used. A graceful fallback.}
\newacro{w3cworldwidewebconsortium}[W3C]{World Wide Web Consortium}
\newacro{htmlhypertextmarkuplanguage}[HTML]{HyperText Markup Language}

 \begin{comment} 



\chapter{Contents}
\label{contents}

\{:.no\_toc\}

\begin{itemize}
\item Will be replaced with the ToC, excluding the ``Contents'' header
\{:toc\}

\end{itemize}

\begin{itemize}
\item Contents (\autoref{contents})

\item Introduction (\autoref{introduction})

\begin{itemize}
\item Organization (\autoref{organization})

\end{itemize}

\item Examples (\autoref{examples})

\begin{itemize}
\item Header (\autoref{header})

\begin{itemize}
\item Header \emph{Containing} \textbf{\emph{Styling}} and a Link (\autoref{headercontainingstylingandalink})

\item Header Containing Attributes \{\#identifier .class1 key=value1\} (\autoref{headercontainingattributesidentifier.class1keyvalue1})

\item Header Unnumbered \{-\} (\autoref{headerunnumbered-})

\item Header Unnumbered 2 \{.Unnumbered\} (\autoref{headerunnumbered2.unnumbered})

\item Auto Cross Reference (\autoref{autocrossreference})

\item User defined reference (\autoref{userdefinedreference})

\item Deeper Levels of Headers (\autoref{deeperlevelsofheaders})

\begin{itemize}
\item Header4 (\autoref{header4})

\begin{itemize}
\item Header5 (\autoref{header5})

\begin{itemize}
\item Header6 (\autoref{header6})

\end{itemize}

\end{itemize}

\end{itemize}

\end{itemize}

\item Backslash Escapes (\autoref{backslashescapes})

\item Emphasis (\autoref{emphasis})

\begin{itemize}
\item CriticMarkup (\autoref{criticmarkup})

\end{itemize}

\item Horizontal Rules (\autoref{horizontalrules})

\item Break (\autoref{break})

\item Superscript \& Subscript (\autoref{superscriptsubscript})

\item Smarty Pants (\autoref{smartypants})

\item Abbreviations (PHP Markdown Extra) (\autoref{abbreviationsphpmarkdownextra})

\item Lists (\autoref{lists})

\begin{itemize}
\item Ordered Lists (\autoref{orderedlists})

\item Unordered Lists (\autoref{unorderedlists})

\item Nested Lists (\autoref{nestedlists})

\item Cutoff a List (\autoref{cutoffalist})

\item List Item in a Block (\autoref{listiteminablock})

\item Fancy Lists (\autoref{fancylists})

\end{itemize}

\end{itemize}

\item . one (\autoref{.one})

\item . two (\autoref{.two})

\begin{itemize}
\item Definition Lists (\autoref{definitionlists})

\begin{itemize}
\item Method 1 (\autoref{method1})

\item Method 2 (\autoref{method2})

\end{itemize}

\item Numbered Example Lists (\autoref{numberedexamplelists})

\item Code (\autoref{code})

\begin{itemize}
\item Fenced Code Blocks (\autoref{fencedcodeblocks})

\begin{itemize}
\item Method 1 (\autoref{method1})

\item Method 2 (\autoref{method2})

\item Method 3 (\autoref{method3})

\end{itemize}

\end{itemize}

\item Block-quotes (\autoref{block-quotes})

\begin{itemize}
\item Block-quotes Quoting Codes (\autoref{block-quotesquotingcodes})

\end{itemize}

\item Line Blocks (\autoref{lineblocks})

\item Tables (\autoref{tables})

\begin{itemize}
\item Method 1 (\autoref{method1})

\item Method 2 (\autoref{method2})

\item Method 3 (\autoref{method3})

\item Method 4 (\autoref{method4})

\end{itemize}

\item [ rows.][rows.]

\begin{itemize}
\item Method 5 (\autoref{method5})

\item Method 6 (\autoref{method6})

\end{itemize}

\item Links (\autoref{links})

\begin{itemize}
\item Reference Links (\autoref{referencelinks})

\end{itemize}

\item Footnotes (\autoref{footnotes})

\begin{itemize}
\item Reference Footnotes (\autoref{referencefootnotes})

\item Glossaries (\autoref{glossaries})

\item Citations (\autoref{citations})

\begin{itemize}
\item MultiMarkdown (\autoref{multimarkdown})

\item Pandoc (\autoref{pandoc})

\end{itemize}

\end{itemize}

\item Images (\autoref{images})

\begin{itemize}
\item Reference Images (\autoref{referenceimages})

\item Image with Links by Nesting Image and Link (\autoref{imagewithlinksbynestingimageandlink})

\item Block Level Images (\autoref{blocklevelimages})

\end{itemize}

\item RAW (\autoref{raw})

\begin{itemize}
\item HTML (\autoref{html})

\item LaTeX (\autoref{latex})

\end{itemize}

\end{itemize}

\item Other Syntaxes (\autoref{othersyntaxes})

\begin{itemize}
\item Metadata (\autoref{metadata})

\begin{itemize}
\item MultiMarkdown Metadata Block (\autoref{multimarkdownmetadatablock})

\item Pandoc Title Block (\autoref{pandoctitleblock})

\item YAML Metadata Block (\autoref{yamlmetadatablock})

\end{itemize}

\item [```][]

\item [It consists of two paragraphs.][itconsistsoftwoparagraphs.]

\item [{\ldots}][...]

\end{itemize}

\item Contents (\autoref{contents})

\begin{itemize}
\item Math (\autoref{math})

\begin{itemize}
\item Markdown (\autoref{markdown})

\item MultiMarkdown and Pandoc (\autoref{multimarkdownandpandoc})

\begin{itemize}
\item MultiMarkdown (\autoref{multimarkdown})

\item Pandoc (\autoref{pandoc})

\end{itemize}

\item Inline Math (\autoref{inlinemath})

\item Block Math (\autoref{blockmath})

\item Other Examples (\autoref{otherexamples})

\end{itemize}

\item File Transclusion (\autoref{filetransclusion})

\end{itemize}

\item References (\autoref{references})

\end{itemize}

 \end{comment} 

\chapter{Introduction}
\label{introduction}

Examples (\autoref{examples}) shows explicit examples for different syntaxes. Other Syntaxes (\autoref{othersyntaxes}) show the syntaxes that can't be shown explicitly.

\section{Organization}
\label{organization}

\begin{itemize}
\item Header levels (except possibly the last header level): features in groups

\item Last header level or a list: different syntaxes

\item TaskPaper-styled tags to indicate in what favor of Markdown such syntax is supported

\begin{itemize}
\item \texttt{@markdown}: supported by original markdown, hence understood to be supported by all variants of markdown

\item \texttt{@ghpages}: GitHub-Favored Markdown, built by kramdown with GFM option. i.e. GitHub Pages' GitHub-Favored Markdown

\item \texttt{@mmd}: MultiMarkdown 

\item \texttt{@pandoc}: pandoc-favored markdown

\item \texttt{@phpextra}: PHP Markdown Extra (inspired some syntax in pandoc and mmd and gfm, not exhaustively tested here)

\end{itemize}

\item TaskPaper Tags

\begin{itemize}
\item \texttt{@...(partial)}: partial supports only

\item \texttt{@...(+...)}: when the extension is used

\item \texttt{@pandoc(-{}-...)}: when the command line argument is used

\item \texttt{@pandoc(parsed)}: not verbatim, but parsed

\end{itemize}

\end{itemize}

Note:

\begin{itemize}
\item You might see \texttt{$<$!-{}- \textbackslash{}begin\{comment\} -{}-$>$...$<$!-{}- \textbackslash{}end\{comment\} -{}-$>$}. This is for mmd to tex to pdf use only. Ignore this.

\end{itemize}

\chapter{Examples}
\label{examples}

\section{Header}
\label{header}

@markdown

See Emphasis (\autoref{emphasis}) and Other Syntaxes (\autoref{othersyntaxes}) to see alternative Setext-style header styles @markdown

\subsection{Header \emph{Containing} \textbf{\emph{Styling}} and a \href{Https://www.wikipedia.org/}{Link}}
\label{headercontainingstylingandalink}

@markdown

\subsection{Header Containing Attributes \{\#identifier .class1 key=value1\}}
\label{headercontainingattributesidentifier.class1keyvalue1}

@pandoc @phpextra

\subsection{Header Unnumbered \{-\}}
\label{headerunnumbered-}

@pandoc

\subsection{Header Unnumbered 2 \{.Unnumbered\}}
\label{headerunnumbered2.unnumbered}

@pandoc

\subsection{Auto Cross Reference}
\label{autocrossreference}

\begin{itemize}
\item Link to Header (\autoref{header}) @pandoc @ghpages @mmd

\item Link to Header (\autoref{header}) @pandoc @mmd

\item Header (\autoref{header}) @mmd @pandoc

\item Header (\autoref{header}) @mmd @pandoc

\end{itemize}

\subsection{User defined reference}
\label{userdefinedreference}

\begin{itemize}
\item userdefinedreference (\autoref{userdefinedreference}) @mmd

\item Link to userdefinedreference (\autoref{userdefinedreference}) @mmd @pandoc(+mmd\_header\_identifiers)

\item Link to ``Header Containing Attributes'' (\autoref{identifier}) @pandoc

\item another-link (\autoref{another-link}) @pandoc

\end{itemize}

\subsection{Deeper Levels of Headers}
\label{deeperlevelsofheaders}

\subsubsection{Header4}
\label{header4}

\paragraph{Header5}
\label{header5}

\subparagraph{Header6}
\label{header6}

\section{Backslash Escapes}
\label{backslashescapes}

*testing* @markdown

\section{Emphasis}
\label{emphasis}

\begin{itemize}
\item \emph{italic} or \emph{italic} @markdown

\item \textbf{bold} or \textbf{bold} @markdown

\item \textbf{\emph{bold italic}} or \textbf{\emph{bold italic}} @markdown

\item \ensuremath{\sim}\ensuremath{\sim}strikethrough\ensuremath{\sim}\ensuremath{\sim} @pandoc

\item Small caps @pandoc @markdown(html)

\end{itemize}

\subsection{CriticMarkup}
\label{criticmarkup}

Visually it looks like emphasis. Functionally it is much more, and called Critic Markup @mmd

\begin{itemize}
\item Deletions from the original text: This is is a test.

\item Additions: This a test.

\item Substitutions: This isn't a test.

\item Highlighting: This is a test.

\item Comments: This is a test.

\end{itemize}

See more at \href{http://fletcher.github.io/MultiMarkdown-5/criticmarkup.html}{CriticMarkup---MultiMarkdown Documentation}\footnote{\href{http://fletcher.github.io/MultiMarkdown-5/criticmarkup.html}{http:/\slash fletcher.github.io\slash MultiMarkdown-5\slash criticmarkup.html}}.

\section{Horizontal Rules}
\label{horizontalrules}

@markdown

\begin{center}\rule{3in}{0.4pt}\end{center}


3 or more hyphens or asterisks

\begin{center}\rule{3in}{0.4pt}\end{center}


\section{Break}
\label{break}

@markdown

No break
like this

Soft break\\
like this

Hard break

like this

\section{Superscript \& Subscript}
\label{superscriptsubscript}

\begin{itemize}
\item x\textsuperscript{2} @mmd

\item d\textsubscript{o} @mmd

\item x\textsuperscript{a+b} @mmd @pandoc

\item x\textsubscript{y,z} @mmd @pandoc

\item P\textsubscript{a}\textbackslash{} cat\ensuremath{\sim} @pandoc

\end{itemize}

\section{Smarty Pants}
\label{smartypants}

@markdown(+smartypants) @pandoc(--smart) @ghpages

\begin{itemize}
\item ``Example 1''

\item `Example 2'

\item en--dash

\item em---dash

\item ellipsis{\ldots}

\end{itemize}

@mmd

\begin{itemize}
\item ``Example 3''

\end{itemize}

\section{Abbreviations (PHP Markdown Extra)}
\label{abbreviationsphpmarkdownextra}

@mmd @phpextra @pandoc(+abbreviations)

Testing abbreviations: \ac{htmlhypertextmarkuplanguage}, \ac{w3cworldwidewebconsortium} (mouseover it to see)

\section{Lists}
\label{lists}

\subsection{Ordered Lists}
\label{orderedlists}

@markdown

\begin{enumerate}
\item test

\item test

\item test

\end{enumerate}

\subsection{Unordered Lists}
\label{unorderedlists}

@markdown

\begin{itemize}
\item test

\item test

\item test

\end{itemize}

\subsection{Nested Lists}
\label{nestedlists}

@markdown

\begin{itemize}
\item test

\begin{itemize}
\item test

\end{itemize}

\item test

\begin{enumerate}
\item test

\item test

\begin{itemize}
\item test

\begin{enumerate}
\item test

\item test

\end{enumerate}

\end{itemize}

\item test

\end{enumerate}

\item test

\end{itemize}

Note about LaTeX output in mmd\slash pandoc:

\begin{itemize}
\item The Maximum nesting level of lists in LaTeX is 4. The quick hack is to mix itemize and enumerate alternatively to go beyond this.

\end{itemize}

\subsection{Cutoff a List}
\label{cutoffalist}

@markdown

\begin{enumerate}
\item one

\item two

\item three

\end{enumerate}

 

\begin{enumerate}
\item uno

\item dos

\item tres

\end{enumerate}

 

\begin{itemize}
\item item one

\item item two

\end{itemize}

 end of list 

\begin{adjustwidth}{2.5em}{2.5em}
\begin{verbatim}

{ my code block }

\end{verbatim}
\end{adjustwidth}

\subsection{List Item in a Block}
\label{listiteminablock}

@markdown

\begin{itemize}
\item First paragraph.

Continued.

\item Second paragraph. With a code block, which must be indented
eight spaces:

\begin{verbatim}
{ code }
\end{verbatim}

\end{itemize}

\subsection{Fancy Lists}
\label{fancylists}

@pandoc

\chapter{. one}
\label{.one}

\chapter{. two}
\label{.two}

9) Ninth
10) Tenth
11) Eleventh
 i. \texttt{i}
 ii. \texttt{ii}
 iii. \texttt{iii}
(2) Two
(5) Three
1. Four
* Five

\section{Definition Lists}
\label{definitionlists}

\subsection{Method 1}
\label{method1}

@mmd @phpextra @pandoc @ghpages

\begin{description}

\item[Physics]

The Fundamental of Science

Describe the Nature

Make Prediction
\end{description}

\subsection{Method 2}
\label{method2}

@ghpages @pandoc @mmd

\begin{description}

\item[Term 1]

Definition 1

\item[Term 2 with \emph{inline markup}]

Definition 2

\begin{verbatim}
{ some code, part of Definition 2 }
\end{verbatim}

Third paragraph of definition 2.
\end{description}

\section{Numbered Example Lists}
\label{numberedexamplelists}

@pandoc

(@) My first example will be numbered (1).
(@) My second example will be numbered (2).

Explanation of examples.

(@) My third example will be numbered (3).

(@good) This is a good example.

As (@good) illustrates, {\ldots}

\section{Code}
\label{code}

\begin{itemize}
\item \texttt{testing} @markdown

\item \texttt{\textbackslash{}[\textbackslash{}ket\{a\}\textbackslash{}]}\{.latex\} @pandoc

\end{itemize}

\subsection{Fenced Code Blocks}
\label{fencedcodeblocks}

\subsubsection{Method 1}
\label{method1}

@markdown

\begin{adjustwidth}{2.5em}{2.5em}
\begin{verbatim}

test
test
    test
    # test

\end{verbatim}
\end{adjustwidth}

\subsubsection{Method 2}
\label{method2}

@markdown(partial:language-not-supported) @ghpages @pandoc @mmd

\begin{adjustwidth}{2.5em}{2.5em}
\begin{lstlisting}[language=tex]
\nabla \times \mathbf{E} = - \frac{\partial \mathbf{B}}{\partial t}

\end{lstlisting}
\end{adjustwidth}

\subsubsection{Method 3}
\label{method3}

@pandoc

\ensuremath{\sim}\ensuremath{\sim}\ensuremath{\sim}markdown
test
test
 test
 \# test
\ensuremath{\sim}\ensuremath{\sim}\ensuremath{\sim}

\ensuremath{\sim}\ensuremath{\sim}\ensuremath{\sim} \{\#mycode .markdown .numberLines startFrom=``100''\}
test
test
 test
 \# test
\ensuremath{\sim}\ensuremath{\sim}\ensuremath{\sim}

\section{Block-quotes}
\label{block-quotes}

@markdown

\begin{quote}

\subsubsection{Test}
\label{test}

test

\begin{quote}

test

test
\end{quote}

\begin{itemize}
\item test

\item test

\end{itemize}
\end{quote}

\subsection{Block-quotes Quoting Codes}
\label{block-quotesquotingcodes}

@markdown

\begin{quote}

\begin{verbatim}
\newcommand...
\end{verbatim}
\end{quote}

\section{Line Blocks}
\label{lineblocks}

@ghpages(partial) @pandoc

\textbar{} The limerick packs laughs anatomical
\textbar{} In space that is quite economical.
\textbar{} But the good ones I've seen
\textbar{} So seldom are clean
\textbar{} And the clean ones so seldom are comical

\textbar{} 200 Main St.
\textbar{} Berkeley, CA 94718

\section{Tables}
\label{tables}

\subsection{Method 1}
\label{method1}

@ghpages @pandoc @mmd

\begin{description}

\item[\textbar{} Right \textbar{} Left \textbar{} Default \textbar{} Center \textbar{}]

\item[\textbar{}------:\textbar{}:-----\textbar{}---------\textbar{}:------:\textbar{}]

\item[\textbar{} 12 \textbar{} 12 \textbar{} 12 \textbar{} 12 \textbar{}]

\item[\textbar{} 123 \textbar{} 123 \textbar{} 123 \textbar{} 123 \textbar{}]

\item[\textbar{} 1 \textbar{} 1 \textbar{} 1 \textbar{} 1 \textbar{}]

Notice how pandoc can have optional caption like this
\end{description}

\subsection{Method 2}
\label{method2}

@mmd

\begin{table}[htbp]
\begin{minipage}{\linewidth}
\setlength{\tymax}{0.5\linewidth}
\centering
\small
\caption{Table Caption}
\label{tablecaption}
\begin{tabulary}{\textwidth}{@{}LRC@{}} \toprule
&\multicolumn{2}{r}{Grouping}\\
Left align&Right align&Center align\\
\midrule
This&This&This\\
column&column&column\\
will&will&will\\
be&be&be\\
left&right&center\\
aligned&aligned&aligned\\
\multicolumn{3}{l}{And a big grouping is like this}\\

\bottomrule

\end{tabulary}
\end{minipage}
\end{table}

\subsection{Method 3}
\label{method3}

@pandoc

Right Left Center Default
------- ------ ---------- -------
 12 12 12 12
 123 123 123 123
 1 1 1 1

Table: Demonstration of simple table syntax.

\subsection{Method 4}
\label{method4}

@pandoc

\begin{center}\rule{3in}{0.4pt}\end{center}


Centered Default Right Left
 Header Aligned Aligned Aligned
----------- ------- --------------- -------------------------
 First row 12.0 Example of a row that
 spans multiple lines.

Second row 5.0 Here's another one. Note
 the blank line between

\begin{adjustwidth}{2.5em}{2.5em}
\begin{verbatim}

                                rows.

\end{verbatim}
\end{adjustwidth}

\begin{center}\rule{3in}{0.4pt}\end{center}


Table: Here's the caption. It, too, may span
multiple lines.

\subsection{Method 5}
\label{method5}

\begin{description}

\item[@pandoc]

Sample grid table.
\end{description}

+---------------+---------------+--------------------+
\textbar{} Fruit \textbar{} Price \textbar{} Advantages \textbar{}
+===============+===============+====================+
\textbar{} Bananas \textbar{} \$1.34 \textbar{} - built-in wrapper \textbar{}
\textbar{} \textbar{} \textbar{} - bright color \textbar{}
+---------------+---------------+--------------------+
\textbar{} Oranges \textbar{} \$2.10 \textbar{} - cures scurvy \textbar{}
\textbar{} \textbar{} \textbar{} - tasty \textbar{}
+---------------+---------------+--------------------+ 

\subsection{Method 6}
\label{method6}

@ghpages

 \begin{comment} 

\begin{table}[htbp]
\begin{minipage}{\linewidth}
\setlength{\tymax}{0.5\linewidth}
\centering
\small
\begin{tabulary}{\textwidth}{@{}LLCR@{}} \toprule
-----------------+------------+-----------------+----------------\\
Default aligned&Left aligned&Center aligned&Right aligned\\
\midrule
First body part&Second cell&Third cell&fourth cell\\
Second line&foo&\textbf{strong}&blah\\
Third line&blah&blah&bar\\
-----------------+------------+-----------------+----------------\\
Second body&&&\\
2 line&&&\\
=================+============+=================+================\\
Footer row&&&\\
-----------------+------------+-----------------+----------------\\

\bottomrule

\end{tabulary}
\end{minipage}
\end{table}

 \end{comment} 

See \href{http://kramdown.gettalong.org/syntax.html#tables}{Syntax \textbar{} kramdown}\footnote{\href{http://kramdown.gettalong.org/syntax.html\#tables}{http:/\slash kramdown.gettalong.org\slash syntax.html\#tables}} 

\section{Links}
\label{links}

\begin{itemize}
\item https:\slash \slash www.wikipedia.org @pandoc(+autolink\_bare\_uris)

\item \href{https://www.wikipedia.org}{https:/\slash www.wikipedia.org} @markdown

\item \href{https://www.wikipedia.org}{Wikipedia.org}\footnote{\href{https://www.wikipedia.org}{https:/\slash www.wikipedia.org}} @markdown

\item \href{https://www.wikipedia.org}{Wikipedia.org}\footnote{\href{https://www.wikipedia.org}{https:/\slash www.wikipedia.org}} @markdown

\item \href{mailto:support@github.com}{Mail to GitHub}\footnote{\href{mailto:support@github.com}{mailto:support@github.com}} @markdown

\end{itemize}

\subsection{Reference Links}
\label{referencelinks}

\begin{itemize}
\item \href{https://www.google.com}{Search here}\footnote{\href{https://www.google.com}{https:/\slash www.google.com}} @markdown

\item A \href{image.png}{link}\footnote{\href{image.png}{image.png}} with attributes. @mmd

\item \href{image.png}{Another link to the link above}\footnote{\href{image.png}{image.png}}. @mmd

\item A \href{/foo/bar.html}{link1}\footnote{\href{/foo/bar.html}{\slash foo\slash bar.html}} with attributes. @markdown

\item \href{http://fsf.org}{Another link}\footnote{\href{http://fsf.org}{http:/\slash fsf.org}}. @markdown

\item \href{/bar#special}{link3}\footnote{\href{/bar\#special}{\slash bar\#special}}. @markdown

\end{itemize}

\section{Footnotes}
\label{footnotes}

\begin{itemize}
\item Footnotes \footnote{This is a mmd inline footnote} @mmd

\item Footnotes \^{}[This is a pandoc inline footnote] @pandoc

\end{itemize}

\subsection{Reference Footnotes}
\label{referencefootnotes}

\begin{itemize}
\item Footnotes\footnote{This is a footnote} @markdown

\item Long Footnotes \footnote{Here's one with multiple blocks.

Subsequent paragraphs are indented to show that they
belong to the previous footnote.

\begin{verbatim}
{ some.code }
\end{verbatim}

The whole paragraph can be indented, or just the first
line. In this way, multi-paragraph footnotes work like
multi-paragraph list items.} @pandoc @ghpages @mmd

\end{itemize}

This paragraph won't be part of the note, because it
isn't indented.

\subsection{Glossaries}
\label{glossaries}

A special kind of footnote \newglossaryentry{term }{sort={optional sort key},name={term },description={The actual definition belongs on a new line, and can continue on
just as other footnotes. Note how it fallbacks gracefully in Markdown.}}\glsadd{term }. @mmd

See more at \href{http://fletcher.github.io/MultiMarkdown-5/glossary.html}{Glossary---MultiMarkdown Documentation}\footnote{\href{http://fletcher.github.io/MultiMarkdown-5/glossary.html}{http:/\slash fletcher.github.io\slash MultiMarkdown-5\slash glossary.html}}.

\subsection{Citations}
\label{citations}

It can looks like footnotes in \ac{htmlhypertextmarkuplanguage} output.

\subsubsection{MultiMarkdown}
\label{multimarkdown}

@mmd

\begin{itemize}
\item This is a statement that should be attributed to its source ~\citep[p. 23]{Doe:2006}.

\end{itemize}

~\nocite{notcited}

See more at \href{http://fletcher.github.io/MultiMarkdown-5/citations.html}{Citations---MultiMarkdown Documentation}\footnote{\href{http://fletcher.github.io/MultiMarkdown-5/citations.html}{http:/\slash fletcher.github.io\slash MultiMarkdown-5\slash citations.html}}.

\subsubsection{Pandoc}
\label{pandoc}

@pandoc

Very powerful but complicated. See more at \href{http://pandoc.org/README.html#citations}{Citations---Pandoc Documentation}\footnote{\href{http://pandoc.org/README.html\#citations}{http:/\slash pandoc.org\slash README.html\#citations}}.

\section{Images}
\label{images}

\begin{itemize}
\item \includegraphics[keepaspectratio,width=\textwidth,height=0.75\textheight]{image.png} @markdown

\end{itemize}

\subsection{Reference Images}
\label{referenceimages}

\begin{itemize}
\item \includegraphics[keepaspectratio,width=\textwidth,height=0.75\textheight]{image.png} @markdown

\item \includegraphics[width=40pt,height=40pt]{image.png} @mmd @pandoc(+mmd\_link\_attributes)

\item \includegraphics[keepaspectratio,width=\textwidth,height=0.75\textheight]{image.png}\{\#id .class width=30 height=20px\} @pandoc @phpextra(partial)

\item a reference ![image][ref] with attributes. @pandoc @phpextra(partial)

\end{itemize}

[ref]: image.png ``optional title'' \{\#id .class key=val key2=``val 2''\}

\subsection{Image with Links by Nesting Image and Link}
\label{imagewithlinksbynestingimageandlink}

\begin{itemize}
\item \href{https://www.wikipedia.org/}{\includegraphics[keepaspectratio,width=\textwidth,height=0.75\textheight]{image.png}}\footnote{\href{https://www.wikipedia.org/}{https:/\slash www.wikipedia.org\slash }} @markdown

\end{itemize}

\subsection{Block Level Images}
\label{blocklevelimages}

\begin{itemize}
\item Block level: \ac{htmlhypertextmarkuplanguage} \texttt{figure} element in MultiMarkdown @mmd @pandoc

\end{itemize}

\begin{figure}[htbp]
\centering
\includegraphics[keepaspectratio,width=\textwidth,height=0.75\textheight]{image.png}
\caption{\textbf{\emph{Block}} \textbf{Level}}
\end{figure}

\includegraphics[keepaspectratio,width=\textwidth,height=0.75\textheight]{image.png}


\section{RAW}
\label{raw}

\subsection{HTML}
\label{html}

\begin{itemize}
\item This should \emph{not} be markdown (or is it?)  @markdown

\item This \emph{is} markdown @mmd @pandoc(+markdown\_attribute)

\end{itemize}

See more at \href{http://fletcher.github.io/MultiMarkdown-5/raw.html}{Raw---MultiMarkdown Documentation}\footnote{\href{http://fletcher.github.io/MultiMarkdown-5/raw.html}{http:/\slash fletcher.github.io\slash MultiMarkdown-5\slash raw.html}}. See test in [Babelmark 2 - Compare markdown implementations](http:\slash \slash johnmacfarlane.net\slash babelmark2\slash ?normalize=1\&text=\%3Cdiv\%3EThis+should+\emph{not}+be+markdown+(or+is+it\%3F\%29+\%3C\%2Fdiv\%3E\%0A\%3Cdiv+markdown\%3D1\%3EThis+\emph{is}+markdown\%3C\%2Fdiv\%3E).

\subsection{LaTeX}
\label{latex}

\begin{itemize}
\item  \newcommand\rawlatex{}  @mmd

\item \textbackslash{}newcommand\textbackslash{}rawlatex\{\} @pandoc(parsed)

\item \textbackslash{}begin\{{\ldots}\} @pandoc

\end{itemize}

\chapter{Other Syntaxes}
\label{othersyntaxes}

\section{Metadata}
\label{metadata}

Note: mmd accepts capitalized metadata keys but others do not. For maximum compatibility, \texttt{author(s)}, \texttt{title}, etc. should be in lower cases.

\subsection{MultiMarkdown Metadata Block}
\label{multimarkdownmetadatablock}

@mmd @pandoc(+mmd\_title\_block)

\begin{adjustwidth}{2.5em}{2.5em}
\begin{verbatim}

title:    A Sample MultiMarkdown Document  
author:   Fletcher T. Penney  
date:     February 9, 2011  
comment:  This is a comment intended to demonstrate  
          metadata that spans multiple lines, yet  
          is treated as a single value.  
CSS:      http://example.com/standard.css

\end{verbatim}
\end{adjustwidth}

See more at \href{http://fletcher.github.io/MultiMarkdown-5/metadata.html}{Metadata---MultiMarkdown Documentation}\footnote{\href{http://fletcher.github.io/MultiMarkdown-5/metadata.html}{http:/\slash fletcher.github.io\slash MultiMarkdown-5\slash metadata.html}}.

\subsection{Pandoc Title Block}
\label{pandoctitleblock}

@pandoc

\begin{adjustwidth}{2.5em}{2.5em}
\begin{verbatim}

% title
% author(s) (separated by semicolons)
% date

\end{verbatim}
\end{adjustwidth}

\subsection{YAML Metadata Block}
\label{yamlmetadatablock}

@Mmd(partial)

@pandoc @ghpages

\begin{adjustwidth}{2.5em}{2.5em}
\begin{verbatim}

---
title:    A Sample MultiMarkdown Document  
author:   Fletcher T. Penney  
date:     February 9, 2011  
tags: [nothing, nothingness]
abstract: |
  This is the abstract.

  It consists of two paragraphs.
---

\end{verbatim}
\end{adjustwidth}

\section{TOC}
\label{toc}

\subsection{Pandoc}
\label{pandoc}

@pandoc

Use \texttt{-{}-toc} as a command argument.

\subsection{MultiMarkdown}
\label{multimarkdown}

@mmd

\texttt{\{\{TOC\}\}}, see beginning. It preprocess the headings and generate a ToC on its own, and doesn't give instruction for LaTeX to generate one. A hack is like this:

\begin{adjustwidth}{2.5em}{2.5em}
\begin{verbatim}

---
...
LaTeX Input:    mmd-load-toc-setcounter
LaTeX Input:    mmd-load-toc
...
---
<!-- \begin{comment} -->
{{TOC}}
<!-- \end{comment} -->
...

\end{verbatim}
\end{adjustwidth}

See more at \href{https://github.com/ickc/peg-multimarkdown-latex-support}{ickc\slash peg-multimarkdown-latex-support: Default support files for generating LaTeX documents with MMD 3 through MMD 5}\footnote{\href{https://github.com/ickc/peg-multimarkdown-latex-support}{https:/\slash github.com\slash ickc\slash peg-multimarkdown-latex-support}}.

\subsection{Kramdown}
\label{kramdown}

@ghpages

\begin{adjustwidth}{2.5em}{2.5em}
\begin{verbatim}

# Contents
{:.no_toc}

* Will be replaced with the ToC, excluding the "Contents" header
{:toc}

\end{verbatim}
\end{adjustwidth}

\section{Math}
\label{math}

MathJax is assumed. MathJax has many configurable options. See \href{http://mathjax.readthedocs.org/en/latest/tex.html}{MathJax TeX and LaTeX Support — MathJax 2.6 documentation}\footnote{\href{http://mathjax.readthedocs.org/en/latest/tex.html}{http:/\slash mathjax.readthedocs.org\slash en\slash latest\slash tex.html}}.

\subsection{Markdown}
\label{markdown}

@markdown

Add the following at the beginning of the document:

\begin{adjustwidth}{2.5em}{2.5em}
\begin{lstlisting}[language=html]
<script type="text/javascript"
    src="https://cdn.mathjax.org/mathjax/latest/MathJax.js?config=TeX-AMS_CHTML-full">
    </script>

\end{lstlisting}
\end{adjustwidth}

MathJax.js is used and any codes within math delimiters are treated as raw \ac{htmlhypertextmarkuplanguage} and to be processed by MathJax.

MathJax delimiter are \texttt{\$\$...\$\$}, \texttt{\textbackslash{}\textbackslash{}(...\textbackslash{}\textbackslash{})} and \texttt{\textbackslash{}\textbackslash{}[...\textbackslash{}\textbackslash{}]} (because an extra \texttt{\textbackslash{}} can be used to escape from MarkDown).

Depending on the markdown parser, extra tricks might be needed to make sure nothing within the math delimiter is treated as markdown (see \href{http://mathjax.readthedocs.org/en/latest/tex.html#tex-and-latex-in-html-documents}{TeX and LaTeX in \ac{htmlhypertextmarkuplanguage} documents — MathJax 2.6 documentation}\footnote{\href{http://mathjax.readthedocs.org/en/latest/tex.html\#tex-and-latex-in-html-documents}{http:/\slash mathjax.readthedocs.org\slash en\slash latest\slash tex.html\#tex-and-latex-in-html-documents}}). \texttt{\$...\$} can be used with MathJax configuration (see \href{http://mathjax.readthedocs.org/en/latest/tex.html#tex-and-latex-math-delimiters}{TeX and LaTeX math delimiters — MathJax 2.6 documentation}\footnote{\href{http://mathjax.readthedocs.org/en/latest/tex.html\#tex-and-latex-math-delimiters}{http:/\slash mathjax.readthedocs.org\slash en\slash latest\slash tex.html\#tex-and-latex-math-delimiters}}).

\subsection{MultiMarkdown and Pandoc}
\label{multimarkdownandpandoc}

There are subtleties how math should be used in HTML+MathJax and LaTeX output from single markdown source. See more in \href{https://github.com/ickc/mathjax-latex-md-mmd-pandoc}{Testing LaTeX Environments Usage in MathJax From Markdown Conversion (including mmd and pandoc)}\footnote{\href{https://github.com/ickc/mathjax-latex-md-mmd-pandoc}{https:/\slash github.com\slash ickc\slash mathjax-latex-md-mmd-pandoc}}.

\subsubsection{MultiMarkdown}
\label{multimarkdown}

@mmd

Add the following metadata at the beginning of the document:

\begin{adjustwidth}{2.5em}{2.5em}
\begin{lstlisting}[language=html]
HTML header:    <script type="text/javascript"
    src="https://cdn.mathjax.org/mathjax/latest/MathJax.js?config=TeX-AMS_CHTML-full">
    </script>

\end{lstlisting}
\end{adjustwidth}

MultiMarkdown math delimiter are \texttt{\$...\$}, \texttt{\$\$...\$\$}, \texttt{\textbackslash{}\textbackslash{}(...\textbackslash{}\textbackslash{})} and \texttt{\textbackslash{}\textbackslash{}[...\textbackslash{}\textbackslash{}]}.

\subsubsection{Pandoc}
\label{pandoc}

@pandoc(--mathjax)

For pandoc, add \texttt{-{}-mathjax} in the command-line argument.

Default math delimiter for pandoc is \texttt{\$...\$}, \texttt{\$\$...\$\$}. Other options are configurable. See more in \href{http://pandoc.org/README.html#non-pandoc-extensions}{Pandoc - Pandoc User’s Guide}\footnote{\href{http://pandoc.org/README.html\#non-pandoc-extensions}{http:/\slash pandoc.org\slash README.html\#non-pandoc-extensions}}.

\subsection{Inline Math}
\label{inlinemath}

\begin{itemize}
\item $1+1$

\item $1 + 1$ @pandoc(+tex\_math\_double\_backslash)

\end{itemize}

\subsection{Block Math}
\label{blockmath}

\begin{itemize}
\item $$R R^T = I$$

\item \[A^T_S = B\]

\end{itemize}

\subsection{Other Examples}
\label{otherexamples}

\begin{itemize}
\item $$x = {-b \pm \sqrt{b^2-4ac} \over 2a}$$

\item \$\$
\textbackslash{}begin\{aligned\}
\textbackslash{}dot\{x\} \& = \textbackslash{}sigma(y-x) \textbackslash{}
\textbackslash{}dot\{y\} \& = \textbackslash{}rho x - y - xz \textbackslash{}
\textbackslash{}dot\{z\} \& = -\textbackslash{}beta z + xy
\textbackslash{}end\{aligned\}
\$\$

\item $$\left( \sum_{k=1}^n a_k b_k \right)^2 \leq \left( \sum_{k=1}^n a_k^2 \right) \left( \sum_{k=1}^n b_k^2 \right)$$

\item $$\mathbf{V}_1 \times \mathbf{V}_2 =  \begin{vmatrix}
\mathbf{i} & \mathbf{j} & \mathbf{k} \\\
\frac{\partial X}{\partial u} &  \frac{\partial Y}{\partial u} & 0 \\\
\frac{\partial X}{\partial v} &  \frac{\partial Y}{\partial v} & 0
\end{vmatrix}$$

\item $$P(E) = {n \choose k} p^k (1-p)^{n-k}$$

\item $$\frac{1}{\Bigl(\sqrt{\phi \sqrt{5}}-\phi\Bigr) e^{\frac25 \pi}} =
1+\frac{e^{-2\pi}} {1+\frac{e^{-4\pi}} {1+\frac{e^{-6\pi}}
{1+\frac{e^{-8\pi}} {1+\ldots} } } }$$

\item \$\$
\textbackslash{}begin\{aligned\}
\textbackslash{}nabla \textbackslash{}times \textbackslash{}vec\{\textbackslash{}mathbf\{B\}\} -\textbackslash{}, \textbackslash{}frac1c\textbackslash{}, \textbackslash{}frac\{\textbackslash{}partial\textbackslash{}vec\{\textbackslash{}mathbf\{E\}\}\}\{\textbackslash{}partial t\} \& = \textbackslash{}frac\{4\textbackslash{}pi\}\{c\}\textbackslash{}vec\{\textbackslash{}mathbf\{j\}\} \textbackslash{}
\textbackslash{}nabla \textbackslash{}cdot \textbackslash{}vec\{\textbackslash{}mathbf\{E\}\} \& = 4 \textbackslash{}pi \textbackslash{}rho \textbackslash{}
\textbackslash{}nabla \textbackslash{}times \textbackslash{}vec\{\textbackslash{}mathbf\{E\}\}\textbackslash{}, +\textbackslash{}, \textbackslash{}frac1c\textbackslash{}, \textbackslash{}frac\{\textbackslash{}partial\textbackslash{}vec\{\textbackslash{}mathbf\{B\}\}\}\{\textbackslash{}partial t\} \& = \textbackslash{}vec\{\textbackslash{}mathbf\{0\}\} \textbackslash{}
\textbackslash{}nabla \textbackslash{}cdot \textbackslash{}vec\{\textbackslash{}mathbf\{B\}\} \& = 0 \textbackslash{}end\{aligned\}
\$\$

\item $$1 +  \frac{q^2}{(1-q)}+\frac{q^6}{(1-q)(1-q^2)}+\cdots =
\prod_{j=0}^{\infty}\frac{1}{(1-q^{5j+2})(1-q^{5j+3})},
\quad\quad \text{for $|q|<1$}.$$

\end{itemize}

\section{File Transclusion}
\label{filetransclusion}

@mmd

See more at \href{http://fletcher.github.io/MultiMarkdown-5/transclusion.html}{File Transclusion---MultiMarkdown Documentation}\footnote{\href{http://fletcher.github.io/MultiMarkdown-5/transclusion.html}{http:/\slash fletcher.github.io\slash MultiMarkdown-5\slash transclusion.html}}.

\chapter{References}
\label{references}

Some examples are directly or indirectly copied from the following documentations:

\begin{enumerate}
\item \href{http://pandoc.org/README.html}{Pandoc - Pandoc User’s Guide}\footnote{\href{http://pandoc.org/README.html}{http:/\slash pandoc.org\slash README.html}}

\item \href{http://fletcher.github.io/MultiMarkdown-5/}{MultiMarkdown User's Guide}\footnote{\href{http://fletcher.github.io/MultiMarkdown-5/}{http:/\slash fletcher.github.io\slash MultiMarkdown-5\slash }}

\item \href{http://kramdown.gettalong.org/syntax.html#tables}{Syntax \textbar{} kramdown}\footnote{\href{http://kramdown.gettalong.org/syntax.html\#tables}{http:/\slash kramdown.gettalong.org\slash syntax.html\#tables}}

\end{enumerate}

\begin{thebibliography}{0}



















\bibitem{Doe:2006}
John Doe. \emph{Some Big Fancy Book}. Vanity Press, 2006. 
\bibitem{notcited}
John Doe. \emph{Another Big Fancy Book}. Vanity Press, 2016. 

\end{thebibliography}

\input{mmd-memoir-footer}

\end{document}
